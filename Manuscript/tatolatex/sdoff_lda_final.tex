% tATOguide.tex
% v2.0 released March 2015

\documentclass{tATO2e}

\usepackage{epstopdf}% To incorporate .eps illustrations using PDFLaTeX, etc.
\usepackage{subfigure}% Support for small, `sub' figures and tables

%\usepackage[longnamesfirst,sort]{natbib}% Citation support using natbib.sty
%\bibpunct[, ]{(}{)}{;}{a}{,}{,}% Citation support using natbib.sty

\usepackage[natbibapa,nodoi]{apacite}% Citation support using apacite.sty. Commands using natbib.sty MUST be deactivated first!

\newcommand{\sdoff}{SD$_{off}$}
\newcommand{\sdon}{SD$_{on}$}
\newcommand{\sddur}{SD$_{dur}$}

\usepackage{caption}


\begin{document}

%\jvol{00} \jnum{00} \jyear{2015} \jmonth{March}

\articletype{}% NOT REQUIRED IN A RESEARCH ARTICLE

\title{\textit{Classification of clustered snow off dates over British Columbia, Canada, from mean sea level pressure}}

\author{
\name{H. E. Gleason$^{1}$, A. R. Bevington$^{1,}$$^{2}$$^{\ast}$\thanks{$^\ast$Corresponding author's email: alexandre.bevington@gov.bc.ca} and V. N. Foord$^{1}$}
\affil{$^{1}$British Columbia Ministry of Forests, Lands, Natural Resource Operations and Rural Development, Prince George, V2L 1R5, Canada}
\affil{$^{2}$Natural Resources and Environmental Studies Institute and Geography Program, University of Northern British Columbia, Prince George, V2N 4Z9, Canada}
\received{v2.0 released March 2015}
}

\maketitle

\begin{abstract}
Atmosphere-ocean teleconnections influence the accumulation and melt of snow in western Canada and can be useful in seasonal forecasting of snowmelt and runoff. Inter-annual variation in these atmosphere-ocean modes has been shown to influence the accumulation and melt of snow within British Columbia (BC), Canada. We investigate fall mean sea level pressure (MSLP) globally as a predictor of remotely sensed snowmelt dates within BC. We use the last day of continuous snow cover (\sdoff{}) detected from timeseries satellite imagery acquired by the Moderate Resolution Imaging Spectroradiometer (MODIS) for the hydrological years 2000--2018. \sdoff{} has been shown to be correlated with continuous snow duration, and is also of interest to seasonal forecasters. Global MSLP from the European Centre for Medium-Range Weather Forecasts ERA5 reanalysis was obtained over hydrological years 1979--2018. S-mode (time vs. location) principal component analysis was carried out on both datasets. The \sdoff{} principle component scores were grouped using a \textit{k}-means clustering routine. Using evolutionary feature selection the subset of MSLP principle components that provided good linear discrimination of the \sdoff{} clusters were found. We explore the atmospheric MSLP principle components that influence the timing of snowmelt over BC, and use them to predict the \sdoff{} clusters at a seasonal lead time. 
\end{abstract}

\begin{keywords}
British Columbia, Canada, MODIS, Snow cover timing, Reanalysis, Sea level pressure, Timeseries
\end{keywords}


\section{Introduction}
In western Canada spatial and annual variation in winter precipitation patterns are highly dependent on synoptic-scale circulation patterns that influence the formation and trajectory of winter storms off the North Pacific Ocean. Global scale atmosphere-ocean modes, or teleconnections, have been shown to influence the frequency of these different synoptic-scale circulation types over BC and western Canada \citep{Stahl2006-og}.
\par
The primary atmosphere-ocean interactions that have been shown to influence winter precipitation in western Canada are the El Niño Southern Oscillation (ENSO), Pacific Decadal Oscillation (PDO), Pacific North American Pattern (PNA) and Arctic Oscillation (AO). Prior studies have related variation in these climate modes to variation in climate over BC, including temperature and precipitation \citep[e.g.,][]{Shabbar1996-oc, Mantua1997-ri} and the timing and duration of snow cover \citep[e.g.,][]{Moore1996-xr,Bevington2019}. 
\par
The Moderate Resolution Imaging Spectroradiometer (MODIS) satellite sensor images the world four times daily as a result of the ascending and descending overpasses from MODIS Terra and MODIS Aqua platforms. Using MODIS derived snow cover \cite{Bevington2019} derived the start (\sdon{}) and end (\sdoff{}) dates of continuous snow coverage over BC and related these metrics to both the ENSO and PDO. Results from that study revealed strong correlations with both ENSO and the PDO, particularly with the \sdoff{} and duration (\sddur{}), and found that the strength of the correlations varied regionally throughout BC, and are influenced by elevation. As the timing and duration of snow cover has important economic, social and ecological implications for BC, the ability to predict these fields at a seasonal lead time would be a great benefit.
\par
Despite strong links between these atmosphere-ocean modes and snow cover, some variation in snow cover over BC is likely to be described by variation in the atmosphere not captured by teleconnection indices alone, such as has been shown for streamflow response in Colorado \citep{Regonda2006}. Global mean sea level pressure (MSLP) reanalysis datasets provide the opportunity to determine what global atmospheric patterns are influencing snow cover in BC. Many studies have shown that low-frequency variability in global MSLP demonstrates gradually evolving modes closely linked to the ENSO phenomenon, and this slow evolution of the atmosphere provides the opportunity for long range predictive skill exploited by many statistical models based on these atmospheric quantities \citep{Latif1998}. 
\par
The objective of this study was to test the suitability of global MSLP averaged during the fall season to better understand and predict spatial patterns and variation in \sdoff{} annually within BC. 


\section{Study Area and Data}

\subsection{Study Area}
Our study area shown in Fig.~\ref{study-area} includes BC with small inclusions of southeastern Alaska and a ~100 km buffer into neighbouring jurisdictions. This is the maximum extent of the \cite{Bevington2019} data. The purpose of using this extended region is that many important watersheds within BC extend just beyond provincial borders. In order to describe the spatial variation in \sdoff{} with consistency, BC Eco-Provinces are referenced in the subsequent sections, and are labelled in Fig. \ref{study-area} \citep{BCEcoProv}.

\begin{figure}
	\begin{center}
		\includegraphics[width=\linewidth]{Figures/study_area}
		\caption{The study area (white dashed line), which extends just beyond the border of British Columbia, and includes the pan-handel of Alaska. The British Columbia Eco-Province regions are labelled.}
		\label{study-area}
	\end{center}
\end{figure}

\subsection{MODIS Snow Cover}
MODIS is a multispectral optical imaging sensor aboard both the Aqua and Terra satellites and provides important observations of land, atmospheric and oceanic processes globally. With 36 spectral bands varying in resolution from 250-1000 m and a combined daily repeat interval MODIS is well equipped for monitoring changes over time (since late 1999) across broad geographic areas. A number of derivative products are available from MODIS Collection 6 (C6), including the daily snow cover data sets from Terra (MOD10A1, \textit{https://nsidc.org/data/mod10a1}) and Aqua (MYD10A1, \textit{https://nsidc.org/data/myd10a1}) which are based on the normalized difference snow index (NDSI) derived from MODIS Level 1B calibrated radiance \citep{Riggs2016}. In \cite{Bevington2019}, these data were used to determine the \sdon{}, \sdoff{} and \sddur{} of continuous snow cover over the study area, and to quantify their response to the ENSO and PDO regionally. The code and gridded output of this work is available online at (\textit{https://github.com/bcgov/ts-rs-modis-snow}), and the years 2000, 2001 and 2018 were additionally processed for this study. In this analysis we only use annual \sdoff{} dates, which are in units of days-since-1-September (DSS). We decided to apply \sdoff{} in this study because the \sdoff{} date is a known indicator of winter-spring runoff and influences spring-summer soil moisture content \citep{Cohen1994}. In addition, \sdoff{} is a driver of northern vegetation phenology which has wide ranging influences, for example, the migration patterns of ungulates \citep{John2020}.

\subsection{European Centre for Medium-Range Weather Forecasts ERA5}
The European Centre for Medium-Range Weather Forecasts (ECMWF) ERA5 atmospheric reanalysis is a global data product with a 31 km resolution and 137 atmosphere levels from the earth's surface up to a height of 80 km \citep{ERA52019}. For the ERA5 mean sea level surface pressure (MSLP) field, data are assimilated from a combination of land, ship and buoy based pressure observations operated by the World Meteorological Organization-Information System (WIS). The ERA5 reanalysis data is available hourly, or monthly. For this study we use the monthly MSLP product for predicting \sdoff{} (\textit{https://www.ecmwf.int/en/forecasts/datasets/reanalysis-datasets/era5}). The assimilation and physical processing for the ERA5 is carried out by the ECMWF’s Integrated Forecast System (IFS), which is described in the European Centre for Medium-Range Weather Forecasts. The monthly MSLP data were obtained for the years 1979--2018.

\section{Methods}
\begin{figure}
\begin{center}
	\includegraphics[width=10cm]{"Figures/Flowchart/MODIS_SLP"}
	\caption{Workflow diagram for predicting clustered \sdoff{} from ERA5 mean sea level pressure.}
	\label{flow-chart}
\end{center}
\end{figure}

The processing workflow applied in this study is illustrated by Fig.~\ref{flow-chart}. Principle component analysis (PCA) was applied to both the gridded annual \sdoff{} data from \cite{Bevington2019} and the ERA5 monthly MSLP data averaged over the fall months (Sep--Nov). All of the principle component (PC) scores derived from the \sdoff{} time series were then used as inputs into a \textit{k}-means cluster analysis. Decomposition of the \sdoff{} time series was only necessary for computational efficiency and repeatability of the \textit{k}-means cluster analysis, the spatial loadings are presented in Appendix \ref{ApdxA}. The leading PC scores from the MSLP time series served as candidate predictors in linear discriminant analysis (LDA) where the \sdoff{} \textit{k}-means cluster results defined groups. Using the leave-one-out cross-validated (LOOCV) overall accuracy as a cost metric, feature selection for predictive LDA was carried out using an evolutionary approach. For the subset LDA, standardized coefficients were related back to the spatially reconstructed MSLP loadings to help infer atmospheric mechanisms contributing to the discrimination of the \sdoff{} clusters.   

\subsection{K-means cluster analysis of annual \sdoff{}}\label{kmeans}

\subsubsection{Principal component analysis}\label{sdoffpca}
Decomposition of the MODIS derived annual \sdoff{} dates over BC was performed using PCA before applying a \textit{k}-means cluster analysis in order to allow efficient computation. The type of PCA applied was in the form of S-Mode (time vs. location) outlined in \cite{Richman1986}, as this type of decomposition has been extensively applied in the field of atmospheric science and similar disciplines for studying major modes of temporal variation in spatially distributed attribute fields, such as timeseries of mean sea level pressure \citep{Urska2013}. In S-Mode PCA, features represent locations (pixels), while observations correspond to repeated measures (years) (Fig.\ref{flow-chart}). Before applying the PCA, for each location (pixel) fall MSLP data (September-November, see
subsection \ref{sec:3.b}) was first zero-centred by subtracting the 1979-2018 climatological mean from each
observation and then scaled to have unit variance by dividing by the standard deviation. A unrotated PCA solution for both MSLP and \sdoff{} was computed using the R programming language 'stats' package \citep{r_stats}.

\subsubsection{K-means cluster analysis}
The \sdoff{} PC scores were used as inputs into a non-hierarchical \textit{k}-means clustering routine. Cluster analysis is by nature an exploratory analysis technique and there are many different clustering methods available \citep{Gotelli2013}. The \textit{k}-means is one of the most frequently applied clustering methods in environmental sciences, and an efficient version detailed in \cite{Hartigan1979} was applied in this study. The \textit{k}-means clustering routine is an iterative method that begins by partitioning the data into \textit{k} groups, forming \textit{k} initial group centres. The algorithm seeks to then find a \textit{k}-partition of the data with a locally optimal within-cluster sum of squares by moving observations from one cluster to another based on their dissimilarity score, typically euclidean distance \citep{Hartigan1979}. Various methods exist for determining the number of \textit{k}-clusters to use, in this study silhouette plots were used to determine an appropriate number of clusters given the \sdoff{} data. Silhouette values represent how similar an observation is to its assigned cluster compared to the other clusters based on multivariate distance, and provide a graphical means for determining cluster appropriateness \citep{Rousseeuw1987}.

\subsection{Decomposition of fall mean sea level pressure}\label{sec:3.b}
The monthly data from the ERA5 MSLP reanalysis were averaged over the fall (Sep--Nov) season preceding the snow accumulation period at each location (pixel) for a given year, referred to here as $MSLP_{FALL}$. The $MSLP_{FALL}$ was calculated for each year over the period 1979--2018. The $MSLP_{FALL}$ time series was transformed though decomposition using methods outlined in Section \ref{sdoffpca}. Common to S-Mode PCA analysis is the visual interpretation of the spatially reconstructed  eigenvectors for each of the retained PCs \citep{Urska2013}. The $MSLP_{FALL}$ eigenvectors, also referred to as loadings, were spatially reconstructed in this way to aid in the interpretation of potential underlying physical mechanisms associated with a given $MSLP_{FALL}$ PC. 

\subsection{Linear Discriminant Analysis}
Linear discriminant analysis has multiple origins, both \cite{Fisher1936} and \cite{Mahalanobis1936} developed different approaches to discriminating between groups. There is however a mathematical relationship between Fisher’s linear discriminant functions (LDF) and the classification functions defined by Mahalanobis \citep[e.g.,][]{Kshirsagar1975}. Unlike cluster analysis, discriminant analysis assigns samples to groups or classes. Intuitively LDA is similar to PCA in that they both decompose the data to fewer dimensions. Where PCA is seeking to generate new axis that maximizes the variation within the data, LDA seeks to generate a new axis that maximizes the separation between established categories, and is therefore supervised. Mathematically, a linear combination of the original variables can be represented by Equation \ref{eq1},
\begin{equation}
	Z_{i} = a_{i1}Y_{1} + a_{i2}Y_{2} + ... + a_{in}Y_{n}
	\label{eq1}
\end{equation}

Assuming the multivariate means $\mathbf{\bar{Y}}$ vary among groups, the goal of LDA is to find $\mathbf{a}$ in Equation \ref{eq1} that maximizes the among-group sum of squares for a given set of $Z_{i}$'s \citep{Gotelli2013}. Fisher’s approach proceeds by performing an eigen analysis of $\mathbf{W^{-1}B}$, where $\mathbf{B}$ is the between-group sum of squares cross-products (SSCP) matrix and $\mathbf{W}$ is the within-group SSCP matrix. Ordered by the resulting eigenvalues, the corresponding eigenvectors $\bm{a_{i}}$ are the coefficients in Equation \ref{eq1}, also referred to as canonical discriminant function coefficients, or loadings. When the data is standardized, these loadings indicate which of the discriminating variables are most important in differentiating between the groups. Classification can be performed by minimizing the Mahalanobis distance among observations within groups adjusted for the prior probabilities when they are not equal \citep{Venables2002}. In this study, the $MSLP_{FALL}$ PCs constituted the possible predictor variables, and the \sdoff{} \textit{k}-means clusters served as factors to be classified. Proportional priors were assumed, and linear discriminant analysis was computed using the R programming language \textit{MASS} package \citep{Venables2002}.


\subsection{Evolution feature selection}
A genetic algorithm (GA) is a stochastic optimization and search algorithm inspired by biological evolution and natural selection \citep{Mitchell1998}. A typical application of binary GAs in statistical modelling is subset selection \citep{Scrucca2013}. Feature selection is a natural extension of GAs using a binary string indicating the presence or absence of a predictor from an given candidate subset. The fitness of a candidate subset is usually represented by one of several model selection criteria, such as Akaike information criterion (AIC) or Bayesian information criterion (BIC), in the case of regression \citep{Bozdogan2003}. With the retained $MSLP_{FALL}$ PCs defining the predictor set, and the \sdoff{} clusters as the response variable, the subset of PCs that maximized the leave-one-out cross-validated (LOOCV) overall accuracy from a predictive LDA was found using a GA algorithm, applying principles described in \cite{Scrucca2013}. Specifically, the fitness of a candidate subset was defined as the LOOCV accuracy, where accuracy is determined through resampling. In the case that more than one model had the same LOOCV accuracy, the ratio of the distance between group means to the total variance of each group, in addition to the number of inputs were considered in identifying the most parsimonious model. Generations were continued so that the population mean fitness appeared to become asymptotic, suggesting no more gain in model fitness would be achieved through additional generations.  

\section{Results}
\subsection{Annual \sdoff{} date decomposition}
Principle component analysis of the annual MODIS derived \sdoff{} time series data was applied before cluster analysis to improve computational efficiency and repeatability of the \textit{k}-means clustering. All of the resulting PC scores were used as inputs into \textit{k}-means cluster analysis. The corresponding \sdoff{} spatial loadings are available in Appendix \ref{ApdxA}, and lend insight into the major modes of \sdoff{} variation over BC.   

\subsection{K-means \sdoff{} clusters}

\textit{k}-means cluster analysis was carried out on the PC scores from the PCA of the \sdoff{} time series. Using \textit{k} equal to 2, 3 and 4 clusters, silhouette plots were generated to help determine the most appropriate number of clusters given the \sdoff{} data. These results are illustrated in Fig. \ref{fig:sillhouettes}. Based on the silhouette scores, it was determined that \textit{k=2} clusters was the most appropriate number of clusters, and had the greatest average silhouette width (Fig. \ref{sil_k2}). Table \ref{tab:sdoff_k_ts} shows the \sdoff{} annual time series cluster results for \textit{k=2} clusters. 


\begin{figure}[!h]
	\begin{center}
		\subfigure[]{
			\resizebox*{0.46\textwidth}{!}{\includegraphics{Figures/SDoff/sil_k_2}\label{sil_k2}}}\hspace{5pt}
		\subfigure[]{
			\resizebox*{0.46\textwidth}{!}{\includegraphics{Figures/SDoff/sil_k_3}\label{sil_k3}}}\hspace{5pt}
		\subfigure[]{
			\resizebox*{0.46\textwidth}{!}{\includegraphics{Figures/SDoff/sil_k_4}\label{sil_k4}}}\hspace{5pt}
		\caption{Silhouette plots corresponding to \textit{k}-clusters for the \sdoff{} time series data for 2--4 clusters (x-axis), red line indicates the average silhouette score.}
		\label{fig:sillhouettes}
	\end{center}
\end{figure}


% latex table generated in R 3.6.3 by xtable 1.8-4 package
% Thu Jun 18 10:42:48 2020
\begin{table}[ht]
	\centering
	\begin{tabular}{r|rrrrrrrrrrrrrrrrrrr}
		\hline
		Year & 2000 & 2001 & 2002 & 2003 & 2004 & 2005 & 2006 & 2007 & 2008 & 2009 \\ 
		\hline
		Cluster & 1 & 1 & 2 & 2 & 2 & 1 & 1 & 1 & 1 & 2\\
		\hline
		\hline
		Year & 2010 & 2011 & 2012 & 2013 & 2014 & 2015 & 2016 & 2017 & 2018 \\
		\hline
		Cluster & 1 & 1 & 1 & 1 & 2 & 2 & 2 & 2 & 2 \\ 
		\hline
	\end{tabular}
	\caption{The \sdoff{} time series cluster value from \textit{k}-means cluster analysis with \textit{k=2} for each hydrologic year over the period 2000--2018.} 
	\label{tab:sdoff_k_ts}
\end{table}

From the clustered \sdoff{} time series shown in Tab. \ref{tab:sdoff_k_ts} the average \sdoff{} was calculated over the years corresponding to each cluster, as well as over the entire time series. Each \sdoff{} cluster average is displayed in Fig. \ref{fig:clust_means} as the difference from the \sdoff{} time series mean. In addition, the difference between Cluster 1 and Cluster 2 means is shown. 


\begin{figure}[!h]
	\begin{center}
		\subfigure[Time Series Mean]{
		\resizebox*{0.46\textwidth}{!}{\includegraphics{Figures/SDoff/ts_mean}\label{fig:clust_mean}}}\hspace{5pt}
		\subfigure[Cluster 1]{
			\resizebox*{0.46\textwidth}{!}{\includegraphics{Figures/SDoff/cluster1_mean}\label{fig:clust_1}}}\hspace{5pt}
		\subfigure[Cluster 2]{
			\resizebox*{0.46\textwidth}{!}{\includegraphics{Figures/SDoff/cluster2_mean}\label{fig:clust_2}}}\hspace{5pt}
		\subfigure[Cluster Difference]{
			\resizebox*{0.46\textwidth}{!}{\includegraphics{Figures/SDoff/clust_diff}\label{fig:clust_diff}}}\hspace{5pt}
		\caption{Difference of \sdoff{} by cluster from the time series average \sdoff{} (a) (2000--2018), and difference between Cluster 1 mean (b) and Cluster 2 mean (c) \sdoff{}. See Table \ref{tab:sdoff_k_ts} for corresponding cluster years.}
		\label{fig:clust_means}
	\end{center}
\end{figure}

The mapped clusters shown in Fig. \ref{fig:clust_means}  indicate that mean \sdoff{} over Cluster 1 years is generally later than the mean \sdoff{} over the entire time series (Fig. \ref{fig:clust_1}), while mean \sdoff{} over Cluster 2 years is generally earlier than mean \sdoff{} over the entire time series (Fig. \ref{fig:clust_2}). This suggests the clusters have classified the \sdoff{} time series into early and late years relative to the time series mean \sdoff{}. The difference between the clusters, represented in Fig. \ref{fig:clust_diff} as Cluster 1 minus Cluster 2, reveals important spatial patterns. A median \sdoff{} difference of 14 days with and interquartile range of 11 days is observed in Fig. \ref{fig:clust_diff}. It appears that along the Coast and Mountains, Southern Alaskan and Georgia Depression regions there are the largest differences in \sdoff{} of around 15--30 days between Cluster 1 and Cluster 2 years. The northwest Central Interior and southwest Southern Interior Mountains also reveal relatively large differences in \sdoff{} between cluster years. Differences in \sdoff{} for Cluster 1 and Cluster 2 years appear less pronounced within some areas of the Northeast Plains and Southern Interior and Mountains regions.  

\subsection{Fall mean sea level pressure decomposition}\label{sec:mslp_decomp}
The total variance explained by each PC derived from PCA of the $MSLP_{FALL}$ time series (1979--2018) is shown in Fig. \ref{fig:mslpscree}. Based on this on this plot, the leading 15 PCs were retained and represent the candidate inputs for LDA feature selection. The first 15 PCs explain 81\% of the total $MSLP_{FALL}$ variance. 

\begin{figure}
	\begin{center}
		\includegraphics[width=10cm]{Figures/MSLP/mslp_scree}
		\caption{Proportion of variance explained for each of the eigenvectors from principle component analysis of $MSLP_{FALL}$ time series over the period 1979--2018. Horizontal line indicates the percent of total variance explained by the 15\textsuperscript{th} PC (1.7\%), PCs below this were not retained in LDA.}
		\label{fig:mslpscree}
	\end{center}
\end{figure}


\subsection{Evolutionary Feature Selection}\label{ga_feat_selec}
The leading PCs described in Section \ref{sec:mslp_decomp} comprised the set of candidate predictors for predictive LDA of the \sdoff{} clusters. Using principles described in \cite{Mitchell1998} the subset of $MSLP_{FALL}$ PC inputs that maximized the LOOCV accuracy of the LDA classifier were identified. With an initial population of 2000 randomly initialized candidate subsets, the initial population was evolved over a total of 150,0000 generations. The population mean LOOCV with each generation is plotted in Fig. \ref{fig:GA_pop_fit}. The absence of change seen in generations beyond 50,0000 suggests an optimal subset of features for classifying the \sdoff{} clusters was likely identified. There were several candidate feature subsets that obtained a LOOCV accuracy of 0.89, which was the maximum observed accuracy of any subset. However, the subset including $MSLP_{FALL}$ PC1, PC2, PC3, PC8, PC10 and PC11 had a high degree of separation with respect to LD1, but significantly fewer inputs than some of the other models with marginally greater separation and was chosen as the best feature subset. The loadings for the subset of $MSLP_{FALL}$ PCs selected for LDA are represented spatially in Appendix B.  


\begin{figure}
	\begin{center}
		\includegraphics[width=10cm]{Figures/LDA/GA_PopAvgLOOCV_TS.jpeg}
		\caption{Mean population fitness (LOOCV accuracy) of LDA classifier over 150,0000 consecutive generations for an initial population size of 2000 randomly assigned $MSLP_{FALL}$ PC candidate feature subsets.}
		\label{fig:GA_pop_fit}
	\end{center}
\end{figure}


\subsection{Linear Discriminant Analysis}
The subset of $MSLP_{FALL}$ PCs described in \ref{ga_feat_selec} were used as input in LDA for predicting the \sdoff{} clusters. The $MSLP_{FALL}$ PCs are shown in Fig. \ref{PC_BoxPlt} by \sdoff{} cluster showing the relative separation each PC provides. Form these plots, we see that $MSLP_{FALL}$ PC1 seems to provide the most definitive separation of the \sdoff{} clusters. 

\begin{figure}
	\begin{center}
		\subfigure[$MSLP_{FALL}$ PC1]{
			\resizebox*{0.24\textwidth}{!}{\includegraphics{Figures/LDA/PC1_Box}\label{PC1_Box}}}\hspace{5pt}
		\subfigure[$MSLP_{FALL}$ PC2]{
			\resizebox*{0.24\textwidth}{!}{\includegraphics{Figures/LDA/PC2_Box}\label{PC2_Box}}}\hspace{5pt}
		\subfigure[$MSLP_{FALL}$ PC3]{
			\resizebox*{0.24\textwidth}{!}{\includegraphics{Figures/LDA/PC3_Box}\label{PC3_Box}}}\hspace{5pt}
		\subfigure[$MSLP_{FALL}$ PC8]{
			\resizebox*{0.24\textwidth}{!}{\includegraphics{Figures/LDA/PC8_Box}\label{PC8_Box}}}\hspace{5pt}
		\subfigure[$MSLP_{FALL}$ PC10]{
			\resizebox*{0.24\textwidth}{!}{\includegraphics{Figures/LDA/PC10_Box}\label{PC10_Box}}}\hspace{5pt}
		\subfigure[$MSLP_{FALL}$ PC11]{
			\resizebox*{0.24\textwidth}{!}{\includegraphics{Figures/LDA/PC11_Box}\label{PC11_Box}}}\hspace{5pt}
		\caption{Boxplots showing the distribution of each $MSLP_{FALL}$ PC selected for LDA by \sdoff{} cluster.}
		\label{PC_BoxPlt}
	\end{center}
\end{figure}

The linear discriminant coefficients and function scores are summarized in the following sections. 

\subsubsection{Discriminant function coefficients}
The first discriminant function (LD1) provides a linear transformation of the $MSLP_{FALL}$ subset of PCs into one dimension that has maximal separation between \sdoff{} group means. This linear transformation is defined by the coefficients $\bm{a_{i}}$ in Equation \ref{eq1}. The original LD1 coefficients are shown in Tab. \ref{tab:lda_coef}, and the standardized coefficients (loadings) are shown in Fig. \ref{fig:loadingplot}. The standardized loadings can be used to determine which of the PC inputs are most important in discriminating between the \sdoff{} clusters. Coefficients close to the origin (zero) have less influence on LD1 that those with coefficients larger (in absolute value) than zero. 

% latex table generated in R 3.6.3 by xtable 1.8-4 package
% Mon Jun 22 13:49:45 2020
\begin{table}[ht]
	\centering
	\begin{tabular}{rrrrrrr}
		\hline
		& PC1 & PC2 & PC3 & PC8 & PC10 & PC11 \\ 
		\hline
		LD1 Coef. & -0.00369 & -0.00152 & -0.00232 & 0.00350 & -0.00254 & -0.00402 \\ 
		\hline
	\end{tabular}
	\caption{Discriminate function coefficients (loadings) for first (and only) linear discriminate function (LD1).} 
	\label{tab:lda_coef}
\end{table}


\begin{figure}
	\begin{center}
		\includegraphics[width=10cm]{Figures/LDA/LD1_LoadingPlt.jpeg}
		\caption{Standardized loadings (coefficients) for the first linear discriminant function (LD1). Inputs are the subset of $MSLP_{FALL}$ PCs.}
		\label{fig:loadingplot}
	\end{center}
\end{figure}

The standardized loadings shown in Fig. \ref{fig:loadingplot} suggest that LD1 is contrasting PC1, PC2, PC3, PC10 and PC11 which all have negative standardized coefficients to PC8 which has a positive coefficient. Based on the magnitude of the standardized coefficients, we find $MSLP_{FALL}$ PC1 has the greatest influence on LD1, while PC10 has the least. 


\subsubsection{Discriminant function scores}
We use a scatterplot of the two \sdoff{} clusters projected onto the dimension of the LD1 discriminant
solution to assess the separation of the \sdoff{} clusters (Fig. \ref{fig:ld1_scoreplot}). From this plot it appears that observations with LD1 scores less than zero indicate \sdoff{} Cluster 1 years, while LD1 scores greater than zero appear to correspond with \sdoff{} Cluster 2. LD1 provides good separation of the two \sdoff{} clusters, the greatest uncertainty occurs for LD1 scores close to the mean of zero, where the probability densities slightly overlap.  

\begin{figure}
	\begin{center}
		\includegraphics[width=10cm]{Figures/LDA/LD1_ScorePlt}
		\caption{Distribution of LD1 scores by \sdoff{} cluster. Dashed line indicates the LD1 mean.}
		\label{fig:ld1_scoreplot}
	\end{center}
\end{figure}





\subsubsection{Leave-one-out cross-validated classification accuracy}
 With a total of 19 observations, the predictive LDA obtained a LOOCV accuracy of 0.89\%, misidentifying two of the observations (Tab. \ref{loocv}). Year 2001 was classified as a Cluster 2 year when in fact it was a Cluster 1 year. Year 2014 was classified as a Cluster 1 year when it was actually Cluster 2. 

\begin{table}
	\tbl{The leave-one-out cross validated confusion matrix for predictive LDA clasification of the \sdoff{} clusters.}
	{\begin{tabular}[l]{@{}lcccc}\toprule
			& \multicolumn{2}{c}{Classified} &  \\
			True Cluster & Cluster 1 & Cluster 2 & Total \\
			\colrule
				Cluster 1 & 9 & 1 & 10 \\
				Cluster 2 & 1 & 8 & 9 \\
			\botrule
	\end{tabular}}
	\label{loocv}
\end{table}

For both 2001 and 2014 it is worth noting neutral ENSO conditions occurred leading up to and during the $MSLP_{FALL}$ period based on the Oceanic Ni\~{n}o Index (ONI) index \citep{Huang2017-ut}. This possibly explains why the LD1 scores for these years were close to the mean, and inaccurately classified.

\section{Discussion}

\subsection{ERA5 Mean Sea Level Pressure and \sdoff{}}
The standardized LDA coefficients reveal which of the $MSLP_{FALL}$ PCs are most influential on the classification of the \sdoff{} clusters (Fig. \ref{fig:loadingplot}). The $MSLP_{FALL}$ PC1 has the greatest influence on LD1 as it has the largest (in absolute value) standardized coefficient, it also explains the most variation in the $MSLP_{FALL}$ time series (Fig. \ref{fig:mslpscree}). The negative LD1 coefficient indicates that as $MSLP_{FALL}$ PC1 increases, LD1 decreases. This means that larger positive PC1 scores are associated with Cluster 1, or relativity late \sdoff{} over BC, while larger negative PC1 scores are associated with Cluster 2, i.e., relativity early \sdoff{} over BC (Fig. \ref{fig:ld1_scoreplot}). The spatial loading pattern for $MSLP_{FALL}$ PC1 displays strong negative loadings centred over Oceania and Southeast Asia, while strong positive loadings occur off the western coast of equatorial South America and the central Pacific Ocean (Fig. \ref{fig:mslp_1_4}). Variation in $MSLP_{FALL}$ at these centres is associated with the Walker circulation, which is closely linked with ENSO \citep{Horel1981}. A common metric of ENSO is the Southern Oscillation Index (SOI), which is the difference in MSLP between Darwin, Australia and Tahiti \citep{Trenberth1976}. $MSLP_{FALL}$ PC1 has strong positive loadings over Tahiti and negative loadings over Darwin and is significantly correlated with the SOI ($\rho = 0.92, p < 0.0001$) and ONI ($\rho = -0.89, p < 0.0001$). This suggest when $MSLP_{FALL}$ is negative the trade winds over the tropical Pacific weaken, leading to a counter current that results in abnormally warm surface sea temperatures in the region. This drives enhanced upper tropospheric divergence in the tropics and convergence in the subtropics, and over  western  Canada this manifests as the the familiar positive PNA pattern \citep{Horel1981}, which has a significant influence on winter climate over western Canada \citep{Shabbar2006-ek}.  Interestingly, PC1 also is significantly correlated ($\rho = 0.35, p = 0.0279$) with the Antarctic Oscillation index (AAO), and the PDO ($\rho = -0.65, p < 0.0001$), however, the physical mechanisms driving these relationships are unclear.
\par
$MSLP_{FALL}$ PC2 has a negative standardized coefficient with LD1 (Fig. \ref{fig:loadingplot}). The spatial loadings for PC2 reveal strong positive loadings throughout Antarctica, while negative loadings occur in the sub-polar regions over the Southern Ocean (Fig. \ref{fig:mslp_1_4}). The loadings for $MSLP_{FALL}$ PC2 indicate a similar pattern to the southern hemisphere annular mode described in \cite{Thompson2000}, and is significantly correlated ($\rho = -0.88, p < 0.0001$) with the AAO. The loadings also reveal strong negative centres over James Bay up to Gulf of Alaska. When $MSLP_{FALL}$ increases over this region,assuming the pattern persists, it is likely that the polar jet stream is diverted by this high pressure, resulting in drier and warmer winter conditions over much of BC \citep{Shabbar2006-ek}. This might explain why positive LD1 values, and earlier \sdoff{}, are associated with negative PC2 scores.
\par
Similar to $MSLP_{FALL}$ PC2, PC3 appears to be contrasting the polar high to the subpolar lows, however, in the northern hemisphere (Fig. \ref{fig:mslp_1_4}). There are wide spread negative loadings spatially corresponding with the northern polar high. Positive loading occur at the subpolar low south of Alaska, and over much of the Pacific and Oceania. This pattern is consistent with the the general loading pattern of the northern hemisphere annular mode \citep{Thompson2000}, and $MSLP_{FALL}$ PC3 is indeed correlated ($\rho = 0.63, p < 0.0001$) with the Arctic Oscillation (AO). While $MSLP_{FALL}$ PC3 does have negative loadings over much of Canada, it also as strong positive loadings centred over the Aleutian Low. Assuming that fall $MSLP_{FALL}$ is indicative of winter patterns, this suggest, when PC3 is positive, the Aleutian Low weakens and likely permits the flow of moist air off the Pacific to be directed away from the Gulf of Alaska and into BC, with increased winter precipitation leading to later \sdoff{} (negative LD1 score).   
\par
The $MSLP_{FALL}$ PC8 is the only PC with a positive LD1 coefficient (Fig. \ref{fig:loadingplot}). Similar to $MSLP_{FALL}$ PC3  positive loadings are observed centred over the Aleutian Low, with negative loadings over much of western Canada (Fig. \ref{fig:mslp_5_6}). However, the negative loadings of $MSLP_{FALL}$ PC8 are more pronounced over western Canada and Alaska. This indicates that PC8 may capture the $MSLP_{FALL}$ pressure over the Canadian High. The positive standardized LD1 coefficient indicates that as $MSLP_{FALL}$ increases over western Canada, Alaska and north Pacific, mean \sdoff{} over BC tends to be later. The enhanced high pressure in this region is associated with the persistence of cold arctic air, and could explain the later \sdoff{} dates. $MSLP_{FALL}$ PC8 has no significant correlation with any established atmosphere-ocean indices (i.e., ONI, SOI, AAO, AO, NAO, PNA or PDO). 
\par
For $MSLP_{FALL}$ PC10 positive loadings are seen over much of western BC (Fig. \ref{fig:mslp_5_6}). The negative LD1 coefficient implies that as $MSLP_{FALL}$ PC10 increases \sdoff{} tends to be later. $MSLP_{FALL}$ PC10 appears to capture variation along the Coast and Mountains of BC, where differences in \sdoff{} between Cluster 1 and Cluster 2 years are large. This is likely do to the fact that when $MSLP_{FALL}$ PC10 is negative, a pronounced low pressure system is located over western BC, which would result in enhanced winter precipitation in the region assuming the pattern persists into the winter months. $MSLP_{FALL}$ PC10 appears to have the least influence on LD1 (Fig. \ref{fig:loadingplot}). 
\par
$MSLP_{FALL}$ PC11 has the most influence on LD1 outside of PC1, but only slightly more influence than the other PCs (Fig. \ref{fig:loadingplot}). The loadings for PC11 show strong positive coefficients with two main centres over the north Pacific Aleutian Islands and off the coast of California (Fig. \ref{fig:mslp_5_6}). There are strong negative coefficients located over the southeast United States and north of Alaska, and moderate negative loading over much of western North America. The negative LD1 coefficient implies that when $MSLP_{FALL}$ is high over western North America \sdoff{} is earlier than average, and when it is high over the North Pacific \sdoff{} is later than average. With higher pressure present over western North America advection of dry continental air towards BC may be enhanced, causing warmer drier conditions over BC, and earlier \sdoff{} dates. $MSLP_{FALL}$ PC11 also appears to be negatively correlated ($\rho = -0.39, p = 0.0133$) with the Pacific North American (PNA) pattern, which has been shown to have an important influence on snow accumulation and melt over BC \citep{Moore1996-xr}.

\subsection{Uncertainties}
While this study has demonstrated that principle components of $MSLP_{FALL}$ can discriminate between clustered patterns of \sdoff{} over BC, the following uncertainties remain. These include how robust these signals are over time. While indices such as ONI evolve at a rate which allows for an annual mean signature, some of the more regionally specific $MSLP_{FALL}$ PCs may change more rapidly. This may suggest changing the aggregation period from the fall period could have a significant impact on the results. Many of the proposed  mechanisms in the study assume the $MSLP_{FALL}$ patterns persist into winter and spring to some degree, however, this persistence may not always occur. In addition, it is common to use field weighting of geophysical data to achieve grid-area parity \citep{Chung1999}. No weighting of $MSLP_{FALL}$ was applied in this study and as a result variability at the poles is likely exaggerated. Finally, while LOOCV was used to establish the out-of-sample accuracy of the predictive LDA, a longer time series would allow for a more robust validation of the overall classification accuracy of the \sdoff{} clusters.

\subsection{Implications}
The cluster analysis of \sdoff{} essentially split the \sdoff{} time series into 'early' and 'late' years relative to the time series mean (Fig. \ref{fig:clust_means}). At locations where the difference between these \sdoff{} clusters is large, there is likely to be significant variation in phenology and hydrology that correspond to the timing of \sdoff{}. The results in this study show that specific PCs of $MSLP_{FALL}$ can be used to discriminate between clusters of remotely sensed \sdoff{} over the previous 19 years when input into a LDA classifier. The spatial loadings for the subset of $MSLP_{FALL}$ PCs help provide insight into some plausible atmospheric mechanisms that explain the ability of each $MSLP_{FALL}$ PC to discriminate between the \sdoff{} clusters over BC. As future satellite data becomes available we will be able to more adequately validate the relationships demonstrated in this study. 



\section{Conclusion}
 We applied MODIS derived 500 m gridded \sdoff{} over BC as input into a \textit{k}-means cluster analysis resulting in two \sdoff{} clusters representing 'early' and 'late' conditions relative to the time series mean \sdoff{} (2000--2018). We aggregated global ERA5 MSLP over the fall season ($MSLP_{FALL}$) annually for the period 1979--2018. Principle component analysis in S-Mode (time vs. location) was applied to the $MSLP_{FALL}$ time series, and the resulting PC scores were joined with the \sdoff{} clusters by year. Using a evolutionary approach, the subset of $MSLP_{FALL}$ PCs that maximized the leave-one-out cross-validated (LOOCV) classification accuracy of the \sdoff{} clusters were identified using a predictive linear discriminate classifier (LDA). For the subset LDA model, an overall predictive LOOCV accuracy of 0.89 was observed, suggesting good discrimination of the \sdoff{} clusters over the time series. The standardized linear discriminate function coefficients were plotted and related to the spatially reconstructed $MSLP_{FALL}$ eigenvectors, or loadings. This comparison allowed us to define the relationship between the subset $MSLP_{FALL}$ loading patterns and atmosphere-ocean modes know to influence the winter climate of western Canada, such as ENSO. However, the $MSLP_{FALL}$ PCs do not represent a standard index of these atmospheric phenomena. Generally, we find the \sdoff{} clusters are sensitive to major variation in the atmosphere, as the leading three $MSLP_{FALL}$ PCs, closely related to ENSO, and the southern and northern annular modes, respectively, were included in the subset LDA. However, several PCs capturing more subtle $MSLP_{FALL}$ variation influencing \sdoff{} were also included in the subset LDA. In locations where the difference in \sdoff{} between clusters is significant this study could have important implications for disciplines where the timing of snow melt is an important forcing, such as the timing of spring freshet and peak flows.
\par
The overall framework of our findings is not likely to change as the dense timeseries of gridded snow cover data continues. The Visible Infrared Imaging Radiometer Suite (VIIRS) satellite sensor aboard the Suomi NPP and NOAA-20 (JPSS1) satellites provides continuity from the MODIS imagery and will allow for further investigation of snow cover over time. In addition, dense timeseries high resolution satellite imagery being collected by small cubsat constellations by private sector companies, such as Planet, may allow for high resolution studies of time series gridded snow cover data. 


\subsection{Acknowledgements}

The authors are very grateful to Michael Allchin, Stephen Déry, Yuexian Wang, John Rex, and William MacKenzie for conversations that helped inform the early direction of this work. The authors thank the reviewers with Atmosphere-Ocean for their valuable guidance provided in response to this work. Also, we acknowledge the open data science community, specifically free access to River Forecast Centre, ERA5, MODIS, Google Earth Engine, R, Python, and QGIS. 

\subsection{Funding}

This research has been supported by the British Columbia Ministry of Forests, Lands, Natural Resource Operations and Rural Development (MFLNRORD grant no. R\&S-28).

\subsection{Supplemental Material}

We will provide the coding in a BC Government GitHub repository once peer-review has been completed. 

\bibliographystyle{apacite}
\bibliography{sdrefs}




\begin{appendices}
\appendix

\section{\sdoff{} Spatial Loadings}\medskip\label{ApdxA}

\begin{figure}
	\begin{center}
		\subfigure[\sdoff{} PC1 Loadings]{
			\resizebox*{0.38\textwidth}{!}{\includegraphics{Figures/SDoff/sdoff_pc1}\label{sdoff_load_1}}}\hspace{5pt}
		\subfigure[\sdoff{} PC2 Loadings]{
			\resizebox*{0.38\textwidth}{!}{\includegraphics{Figures/SDoff/sdoff_pc2}\label{sdoff_load_2}}}\hspace{5pt}
		\subfigure[\sdoff{} PC3 Loadings]{
			\resizebox*{0.38\textwidth}{!}{\includegraphics{Figures/SDoff/sdoff_pc3}\label{sdoff_load_3}}}\hspace{5pt}
		\subfigure[\sdoff{} PC4 Loadings]{
			\resizebox*{0.38\textwidth}{!}{\includegraphics{Figures/SDoff/sdoff_pc4}\label{sdoff_load_4}}}\hspace{5pt}
		\caption{The spatially reconstructed loadings of the leading four eigenvectors from principle component analysis of the \sdoff{} timeseries, years 2000--2018.}
		\label{sdoff_load_1_4}
	\end{center}
\end{figure}

\begin{figure}
	\begin{center}
		\subfigure[\sdoff{} PC5 Loadings]{
			\resizebox*{0.38\textwidth}{!}{\includegraphics{Figures/SDoff/sdoff_pc5}\label{sdoff_load_5}}}\hspace{5pt}
		\subfigure[\sdoff{} PC6 Loadings]{
			\resizebox*{0.38\textwidth}{!}{\includegraphics{Figures/SDoff/sdoff_pc6}\label{sdoff_load_6}}}\hspace{5pt}
		\subfigure[\sdoff{} PC7 Loadings]{
			\resizebox*{0.38\textwidth}{!}{\includegraphics{Figures/SDoff/sdoff_pc7}\label{sdoff_load_7}}}\hspace{5pt}
		\subfigure[\sdoff{} PC8 Loadings]{
			\resizebox*{0.38\textwidth}{!}{\includegraphics{Figures/SDoff/sdoff_pc8}\label{sdoff_load_8}}}\hspace{5pt}
		\caption{The spatially reconstructed loadings of the leading four eigenvectors from principle component analysis of the \sdoff{} timeseries, years 2000--2018.}
		\label{sdoff_load_5_8}
	\end{center}
\end{figure}

\begin{figure}
	\begin{center}
		\subfigure[\sdoff{} PC9 Loadings]{
			\resizebox*{0.38\textwidth}{!}{\includegraphics{Figures/SDoff/sdoff_pc9}\label{sdoff_load_9}}}\hspace{5pt}
		\subfigure[\sdoff{} PC10 Loadings]{
			\resizebox*{0.38\textwidth}{!}{\includegraphics{Figures/SDoff/sdoff_pc10}\label{sdoff_load_10}}}\hspace{5pt}
		\subfigure[\sdoff{} PC11 Loadings]{
			\resizebox*{0.38\textwidth}{!}{\includegraphics{Figures/SDoff/sdoff_pc11}\label{sdoff_load_11}}}\hspace{5pt}
		\subfigure[\sdoff{} PC12 Loadings]{
			\resizebox*{0.38\textwidth}{!}{\includegraphics{Figures/SDoff/sdoff_pc12}\label{sdoff_load_12}}}\hspace{5pt}
		\caption{The spatially reconstructed loadings of the leading four eigenvectors from principle component analysis of the \sdoff{} timeseries, years 2000--2018.}
		\label{sdoff_load_9_12}
	\end{center}
\end{figure}

\begin{figure}
	\begin{center}
		\subfigure[\sdoff{} PC13 Loadings]{
			\resizebox*{0.38\textwidth}{!}{\includegraphics{Figures/SDoff/sdoff_pc13}\label{sdoff_load_13}}}\hspace{5pt}
		\subfigure[\sdoff{} PC14 Loadings]{
			\resizebox*{0.38\textwidth}{!}{\includegraphics{Figures/SDoff/sdoff_pc14}\label{sdoff_load_14}}}\hspace{5pt}
		\subfigure[\sdoff{} PC15 Loadings]{
			\resizebox*{0.38\textwidth}{!}{\includegraphics{Figures/SDoff/sdoff_pc15}\label{sdoff_load_15}}}\hspace{5pt}
		\subfigure[\sdoff{} PC16 Loadings]{
			\resizebox*{0.38\textwidth}{!}{\includegraphics{Figures/SDoff/sdoff_pc16}\label{sdoff_load_16}}}\hspace{5pt}
		\caption{The spatially reconstructed loadings of the leading four eigenvectors from principle component analysis of the \sdoff{} timeseries, years 2000--2018.}
		\label{sdoff_load_13_16}
	\end{center}
\end{figure}

\begin{figure}
	\begin{center}
		\subfigure[\sdoff{} PC17 Loadings]{
			\resizebox*{0.38\textwidth}{!}{\includegraphics{Figures/SDoff/sdoff_pc17}\label{sdoff_load_17}}}\hspace{5pt}
		\subfigure[\sdoff{} PC18 Loadings]{
			\resizebox*{0.38\textwidth}{!}{\includegraphics{Figures/SDoff/sdoff_pc18}\label{sdoff_load_18}}}\hspace{5pt}
		\subfigure[\sdoff{} PC19 Loadings]{
			\resizebox*{0.38\textwidth}{!}{\includegraphics{Figures/SDoff/sdoff_pc19}\label{sdoff_load_19}}}\hspace{5pt}
		\caption{The spatially reconstructed loadings of the leading four eigenvectors from principle component analysis of the \sdoff{} timeseries, years 2000--2018.}
		\label{sdoff_load_17_18}
	\end{center}
\end{figure}



\section{Mean Sea Level Pressure Spatial Loading Scores}\medskip\label{ApdxB}

\begin{figure}
	\begin{center}
		\subfigure[MSLP PC1]{
			\resizebox*{0.48\textwidth}{!}{\includegraphics{Figures/MSLP/mslp_pc1}}}\hspace{5pt}
		\subfigure[MSLP PC2]{
			\resizebox*{0.48\textwidth}{!}{\includegraphics{Figures/MSLP/mslp_pc2}}}\hspace{5pt}
		
		\subfigure[MSLP PC3]{
			\resizebox*{0.48\textwidth}{!}{\includegraphics{Figures/MSLP/mslp_pc3}}}\hspace{5pt}
		\subfigure[MSLP PC8]{
			\resizebox*{0.48\textwidth}{!}{\includegraphics{Figures/MSLP/mslp_pc8}}}\hspace{5pt}
		\caption{Spatially reconstructed PC1, PC2, PC3 and PC8 loadings from PCA of the $MSLP_{FALL}$ time series (1979--2018). Units are dimensionless.}
		\label{fig:mslp_1_4}
	\end{center}
\end{figure}

\begin{figure}
	\begin{center}
		\subfigure[MSLP PC10]{
			\resizebox*{0.48\textwidth}{!}{\includegraphics{Figures/MSLP/mslp_pc10}}}\hspace{5pt}
		\subfigure[MSLP PC11]{
			\resizebox*{0.48\textwidth}{!}{\includegraphics{Figures/MSLP/mslp_pc11}}}\hspace{5pt}
		\caption{Spatially reconstructed PC10 and PC11 loadings from PCA of the $MSLP_{FALL}$ time series (1979--2018). Units are dimensionless.}
		\label{fig:mslp_5_6}
	\end{center}
\end{figure}



\end{appendices}

\end{document}